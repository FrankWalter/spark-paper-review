
\documentclass[conference]{IEEEtran}



\usepackage{amsmath}
\usepackage{amssymb}

\ifCLASSINFOpdf

\else

\fi

\usepackage{graphicx}
\usepackage{listings}

\hyphenation{op-tical net-works semi-conduc-tor}
\begin{document}
	\title{Hadoop-GIS: A High Performance Spatial Data Warehousing System over MapReduce}	
	\maketitle
	\IEEEpeerreviewmaketitle	
	\section{Summary}
	Hadoop-GIS extends Hive, and  is a data warehouse infrastructure build on top of Hadoop with a form grid index for range queries and self-join.
	\section{drawbacks}
	Hadoop-GIS deals with Hadoop as a black box, and hence it remains limited by the limitations of existing Hadoop systems.
	\begin{itemize}
		\item The limitations caused by Hadoop's being ill equipped in supporting spatial data.
		\item Hadoop-GIS can only support uniform grid index.
		\item Users can't define new spatial operations.				
	\end{itemize}
\end{document}