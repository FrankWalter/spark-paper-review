
\documentclass[conference]{IEEEtran}



\usepackage{amsmath}
\usepackage{amssymb}

\ifCLASSINFOpdf

\else

\fi

\usepackage{graphicx}
\usepackage{listings}

\hyphenation{op-tical net-works semi-conduc-tor}
\begin{document}
	
	\title{SJMR: Parallelizing Spatial Join with MapReduce}	
	\maketitle
	\IEEEpeerreviewmaketitle	
	\section{Summary}
	This paper and just introduces a primitive algorithm called SJMR(Spatial Join with MapReduce), without spatial indexes. The following is the detailed description of SJMR algorithm.
	\begin{enumerate}
		\item Determining the partition number. The partition number should make the partitions, which are going to be merged, fit entirely in memory).
		\item Map stage. Redistribute the tuples of R and S into different Reduce tasks according to spatial partitioning function. There is trade-off between Coefficient of variation and replication overhead.
		\item Reduce stage Spatial join is carried out in two steps: filter step and refinement step.
		\begin{enumerate}
			\item Filter step:
			\item Refinement step:
		\end{enumerate}
	\end{enumerate}
	\section{important issues}
	Decomposing the universe into smaller tiles makes it easier
	to produce a more uniform partition distribution. However,
	spatial objects that span tiles from multiple partitions have
	to be replicated in all those partitions, thereby increasing the replication overhead.
	\section{contents require further exploring}
	\begin{itemize}
		\item tile coding method
	\end{itemize}
\end{document}